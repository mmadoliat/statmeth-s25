\documentclass[twoside]{article}
\usepackage{amsgen,amsmath,amstext,amsbsy,amsopn,amssymb,color}
\usepackage{graphicx}
\usepackage{epsfig}

\setlength{\oddsidemargin}{0.1 in} \setlength{\evensidemargin}{-0.1
in} \setlength{\topmargin}{-0.6 in} \setlength{\textwidth}{6.5 in}
\setlength{\textheight}{10.5 in} \setlength{\headsep}{0.1 in}
\setlength{\parindent}{0 in} \setlength{\parskip}{0.1 in}

\newcommand{\red}{\textcolor{red}}

\newcommand{\homework}[2]{
   \pagestyle{myheadings}
   \thispagestyle{plain}
   \newpage
   \setcounter{page}{1}
   \noindent
   \begin{center}
   \framebox{
      \vbox{\vspace{2mm}
       \hbox to 6.28in { {\bf Math 4720:~Statistical Methods \hfill} }
       \vspace{6mm}
       \hbox to 6.28in { {\Large \hfill #1 (#2)  \hfill} }
       \vspace{6mm}
      \vspace{2mm}}
   }
   \end{center}
   \markboth{#1}{#1}
   \vspace*{4mm}
}

\newcommand{\bbF}{\mathbb{F}}
\newcommand{\bbX}{\mathbb{X}}
\newcommand{\bI}{\mathbf{I}}
\newcommand{\bX}{\mathbf{X}}
\newcommand{\bY}{\mathbf{Y}}
\newcommand{\bepsilon}{\boldsymbol{\epsilon}}
\newcommand{\balpha}{\boldsymbol{\alpha}}
\newcommand{\bbeta}{\boldsymbol{\beta}}
\newcommand{\0}{\mathbf{0}}

\begin{document}

\homework{$12^{th}$ Week Summary}{11/26/18}
\vspace{-0.4 in}
\begin{itemize}
\item \textbf{More on ANOVA}\dotfill
\item What if the equality of variances fail?
\subitem We ``transform'' the data
\item What are the common transformations?
\subitem If $\sigma^2\propto\mu$, the use $Y_T=\sqrt{Y}$𝑌or $\sqrt{Y+0.375}$
\subitem If $\sigma^2\propto\mu^2$, the use $Y_T=\ln(Y)$𝑌or $\ln(Y+1)$
\subitem If $\sigma^2\propto\mu(1-\mu)$, the use $Y_T=\sin^{-1}\sqrt{Y}$
\item \textbf{Population proportion}\dotfill
\item Draw a large random sample of size $n$ from a large population having unknown proportion $p$ of successes. To test the hypothesis $H_0: \pi = \pi_0$, compute the $Z$ statistic: $z=\dfrac{\hat{\pi}-\pi_0}{\sqrt{\dfrac{\pi_0(1-\pi_0)}{n}}}$
\item Use this test when the sample size $n$ is so large that both $n\pi_0$ and $n(1 - \pi_0)$ are $5$ or more.
\item The p\_value for a test of $H_0: \pi = \pi_0$ against
\subsubitem $H_a: \pi > \pi_0$ is p-value $ = P(Z\geq z)$
\subsubitem $H_a: \pi < \pi_0$ is p-value $ = P(Z\leq z)$
\subsubitem $H_a: \pi \neq \pi_0$ is p-value $ = 2P(|Z|\geq|z|)$
\item Confidence interval for $\pi$ can be obtained by $\hat{\pi}\pm z_{\alpha/2 v }\sqrt{\dfrac{\hat{\pi}(1-\hat{\pi})}{n}}$.
This interval should not be used unless $n\hat{\pi} \geq 5$ and $n(1-\hat{\pi}) \geq 5$.
\item The \textbf{chi-square goodness-of-fit} test allows us to test whether a categorical variable follows a given probability distribution.
\item A categorical variable has $k$ possible outcomes, with probabilities $\pi_1, \pi_2, \pi_3, \ldots, \pi_k$. That is, $\pi_i$ is the probability of the $i^{th}$ outcome. We have $n$ independent observations from this categorical variable.
\item We use chi-square goodness-of-fit test for testing the null hypothesis that the probabilities have specified values:
\subitem $H_0: \pi_1={\pi_1}_0, \pi_2={\pi_2}_0, \pi_3={\pi_3}_0, \ldots, \pi_k={\pi_k}_0$

\subitem The expected count of outcome $i$, $E_i=n{\pi_i}_0$. Expected counts do not have to be round numbers.
\item The chi-square statistic is a measure of how far observed counts are from expected counts under the null hypothesis. The formula for the statistic is:
\subitem $\chi^2=\sum{\dfrac{(\textrm{observed count - expected count})^2}{\textrm{expected count}}}=\sum_{i=1}^k{\dfrac{(O_i-E_i)^2}{E_i}}$
\subitem Each of the k terms in the sum is called a chi-square component
\item The chi-square goodness of fit test involving $k$ outcomes refers to the chi-square distribution with $k - 1$ degrees of freedom. The p-value is the area to the right of $\chi^2$ under the density curve of this chi-square distribution.
\item You can safely use the chi-square goodness-of-fit test with critical values from the chi-square distribution when no more than $20\%$ of the expected counts are less than $5$ and all individual expected counts are $1$ or greater. In particular, the chi-square goodness-of-fit test can be used when all the expected counts are $5$ or greater.

\end{itemize}

\end{document}


